\documentclass[journal,12pt,twocolumn]{IEEEtran}
%
\usepackage{setspace}
\usepackage{gensymb}

\singlespacing

\usepackage{graphicx}
\graphicspath{{./fig/}}

\usepackage[cmex10]{amsmath}
\usepackage{amsthm}

\usepackage{mathrsfs}
\usepackage{txfonts}
\usepackage{stfloats}
\usepackage{bm}
\usepackage{cite}
\usepackage{cases}
\usepackage{subfig}

\usepackage{longtable}
\usepackage{multirow}

\usepackage{enumitem}
\usepackage{mathtools}
\usepackage{steinmetz}
\usepackage{tikz}
\usepackage{circuitikz}
\usepackage{verbatim}
\usepackage{tfrupee}
\usepackage{inputenc}
\usepackage[breaklinks=true]{hyperref}

\usepackage{tkz-euclide} % loads  TikZ and tkz-base

\usetikzlibrary{calc,math}
\usepackage{listings}
    \usepackage{color}                                            %%
    \usepackage{array}                                            %%
    \usepackage{longtable}                                        %%
    \usepackage{calc}                                             %%
    \usepackage{multirow}                                         %%
    \usepackage{hhline}                                           %%
    \usepackage{ifthen}                                           %%
  %optionally (for landscape tables embedded in another document): %%
    \usepackage{lscape}     
\usepackage{multicol}
\usepackage{chngcntr}
%\usepackage{enumerate}


\DeclareMathOperator*{\Res}{Res}
\renewcommand\thesection{\arabic{section}}
\renewcommand\thesubsection{\thesection.\arabic{subsection}}
\renewcommand\thesubsubsection{\thesubsection.\arabic{subsubsection}}

\renewcommand\thesectiondis{\arabic{section}}
\renewcommand\thesubsectiondis{\thesectiondis.\arabic{subsection}}
\renewcommand\thesubsubsectiondis{\thesubsectiondis.\arabic{subsubsection}}

% correct bad hyphenation here
\hyphenation{op-tical net-works semi-conduc-tor}
\def\inputGnumericTable{}                                 %%

\lstset{
%language=C,
frame=single, 
breaklines=true,
columns=fullflexible
}
%\lstset{
%language=tex,
%frame=single, 
%breaklines=true
%}

\begin{document}
%


\newtheorem{theorem}{Theorem}[section]
\newtheorem{problem}{Problem}
\newtheorem{proposition}{Proposition}[section]
\newtheorem{lemma}{Lemma}[section]
\newtheorem{corollary}[theorem]{Corollary}
\newtheorem{example}{Example}[section]
\newtheorem{definition}[problem]{Definition}
%\newtheorem{thm}{Theorem}[section] 
%\newtheorem{defn}[thm]{Definition}
%\newtheorem{algorithm}{Algorithm}[section]
%\newtheorem{cor}{Corollary}
\newcommand{\BEQA}{\begin{eqnarray}}
\newcommand{\EEQA}{\end{eqnarray}}
\newcommand{\define}{\stackrel{\triangle}{=}}
\bibliographystyle{IEEEtran}
%\bibliographystyle{ieeetr}
\providecommand{\mbf}{\mathbf}
\providecommand{\pr}[1]{\ensuremath{\Pr\left(#1\right)}}
\providecommand{\qfunc}[1]{\ensuremath{Q\left(#1\right)}}
\providecommand{\sbrak}[1]{\ensuremath{{}\left[#1\right]}}
\providecommand{\lsbrak}[1]{\ensuremath{{}\left[#1\right.}}
\providecommand{\rsbrak}[1]{\ensuremath{{}\left.#1\right]}}
\providecommand{\brak}[1]{\ensuremath{\left(#1\right)}}
\providecommand{\lbrak}[1]{\ensuremath{\left(#1\right.}}
\providecommand{\rbrak}[1]{\ensuremath{\left.#1\right)}}
\providecommand{\cbrak}[1]{\ensuremath{\left\{#1\right\}}}
\providecommand{\lcbrak}[1]{\ensuremath{\left\{#1\right.}}
\providecommand{\rcbrak}[1]{\ensuremath{\left.#1\right\}}}
\theoremstyle{remark}
\newtheorem{rem}{Remark}
\newcommand{\sgn}{\mathop{\mathrm{sgn}}}
\providecommand{\abs}[1]{\left\vert#1\right\vert}
\providecommand{\res}[1]{\Res\displaylimits_{#1}} 
\providecommand{\norm}[1]{\left\lVert#1\right\rVert}
%\providecommand{\norm}[1]{\lVert#1\rVert}
\providecommand{\mtx}[1]{\mathbf{#1}}
\providecommand{\mean}[1]{E\left[ #1 \right]}
\providecommand{\fourier}{\overset{\mathcal{F}}{ \rightleftharpoons}}
%\providecommand{\hilbert}{\overset{\mathcal{H}}{ \rightleftharpoons}}
\providecommand{\system}{\overset{\mathcal{H}}{ \longleftrightarrow}}
	%\newcommand{\solution}[2]{\textbf{Solution:}{#1}}
\newcommand{\solution}{\noindent \textbf{Solution: }}
\newcommand{\cosec}{\,\text{cosec}\,}
\providecommand{\dec}[2]{\ensuremath{\overset{#1}{\underset{#2}{\gtrless}}}}
\newcommand{\myvec}[1]{\ensuremath{\begin{pmatrix}#1\end{pmatrix}}}
\newcommand{\mydet}[1]{\ensuremath{\begin{vmatrix}#1\end{vmatrix}}}
%\numberwithin{equation}{section}
\numberwithin{equation}{subsection}
%\numberwithin{problem}{section}
%\numberwithin{definition}{section}
\makeatletter
\@addtoreset{figure}{problem}
\makeatother
\let\StandardTheFigure\thefigure
\let\vec\mathbf
%\renewcommand{\thefigure}{\theproblem.\arabic{figure}}
\renewcommand{\thefigure}{\theproblem}
%\setlist[enumerate,1]{before=\renewcommand\theequation{\theenumi.\arabic{equation}}
%\counterwithin{equation}{enumi}
%\renewcommand{\theequation}{\arabic{subsection}.\arabic{equation}}
\def\putbox#1#2#3{\makebox[0in][l]{\makebox[#1][l]{}\raisebox{\baselineskip}[0in][0in]{\raisebox{#2}[0in][0in]{#3}}}}
     \def\rightbox#1{\makebox[0in][r]{#1}}
     \def\centbox#1{\makebox[0in]{#1}}
     \def\topbox#1{\raisebox{-\baselineskip}[0in][0in]{#1}}
     \def\midbox#1{\raisebox{-0.5\baselineskip}[0in][0in]{#1}}
\vspace{3cm}
\title{Matrix theory - Assignment 8}
\author{Shreeprasad Bhat\\AI20MTECH14011}
%\title{}
% paper title
% can use linebreaks \\ within to get better formatting as desired
%\title{Matrix Analysis through Octave}
%
%
% author names and IEEE memberships
% note positions of commas and nonbreaking spaces ( ~ ) LaTeX will not break
% a structure at a ~ so this keeps an author's name from being broken across
% two lines.
% use \thanks{} to gain access to the first footnote area
% a separate \thanks must be used for each paragraph as LaTeX2e's \thanks
% was not built to handle multiple paragraphs
%
%\author{<-this % stops a space
%\thanks{}}
%}
% note the % following the last \IEEEmembership and also \thanks - 
% these prevent an unwanted space from occurring between the last author name
% and the end of the author line. i.e., if you had this:
% 
% \author{....lastname \thanks{...} \thanks{...} }
%                     ^------------^------------^----Do not want these spaces!
%
% a space would be appended to the last name and could cause every name on that
% line to be shifted left slightly. This is one of those "LaTeX things". For
% instance, "\textbf{A} \textbf{B}" will typeset as "A B" not "AB". To get
% "AB" then you have to do: "\textbf{A}\textbf{B}"
% \thanks is no different in this regard, so shield the last } of each \thanks
% that ends a line with a % and do not let a space in before the next \thanks.
% Spaces after \IEEEmembership other than the last one are OK (and needed) as
% you are supposed to have spaces between the names. For what it is worth,
% this is a minor point as most people would not even notice if the said evil
% space somehow managed to creep in.
% The paper headers
%\markboth{Journal of \LaTeX\ Class Files,~Vol.~6, No.~1, January~2007}%
%{Shell \MakeLowercase{\textit{et al.}}: Bare Demo of IEEEtran.cls for Journals}
% The only time the second header will appear is for the odd numbered pages
% after the title page when using the twoside option.
% 
% *** Note that you probably will NOT want to include the author's ***
% *** name in the headers of peer review papers.                   ***
% You can use \ifCLASSOPTIONpeerreview for conditional compilation here if
% you desire.
% If you want to put a publisher's ID mark on the page you can do it like
% this:
%\IEEEpubid{0000--0000/00\$00.00~\copyright~2007 IEEE}
% Remember, if you use this you must call \IEEEpubidadjcol in the second
% column for its text to clear the IEEEpubid mark.
% make the title area
\maketitle
\newpage
%\tableofcontents
\bigskip
\renewcommand{\thefigure}{\theenumi}
\renewcommand{\thetable}{\theenumi}
%\renewcommand{\theequation}{\theenumi}
%\begin{abstract}
%%\boldmath
%In this letter, an algorithm for evaluating the exact analytical bit error rate  (BER)  for the piecewise linear (PL) combiner for  multiple relays is presented. Previous results were available only for upto three relays. The algorithm is unique in the sense that  the actual mathematical expressions, that are prohibitively large, need not be explicitly obtained. The diversity gain due to multiple relays is shown through plots of the analytical BER, well supported by simulations. 
%
%\end{abstract}
% IEEEtran.cls defaults to using nonbold math in the Abstract.
% This preserves the distinction between vectors and scalars. However,
% if the journal you are submitting to favors bold math in the abstract,
% then you can use LaTeX's standard command \boldmath at the very start
% of the abstract to achieve this. Many IEEE journals frown on math
% in the abstract anyway.
% Note that keywords are not normally used for peerreview papers.
%\begin{IEEEkeywords}
%Cooperative diversity, decode and forward, piecewise linear
%\end{IEEEkeywords}
% For peer review papers, you can put extra information on the cover
% page as needed:
% \ifCLASSOPTIONpeerreview
% \begin{center} \bfseries EDICS Category: 3-BBND \end{center}
% \fi
%
% For peerreview papers, this IEEEtran command inserts a page break and
% creates the second title. It will be ignored for other modes.
%\IEEEpeerreviewmaketitle
\begin{abstract}
This document illustrates finding subspaces of a vector spaces
\end{abstract}
Download latex-tikz from
\begin{lstlisting}
https://github.com/shreeprasadbhat/matrix-theory/blob/master/assignment8/
\end{lstlisting}
%
\section{Problem}
\begin{enumerate}[label=\alph*.]
    \item Prove that only subspace of $\mathbb{R}^1$ are $\mathbb{R}^1$ and the zero subspace 
    \item Prove that a subspace of $\mathbb{R}^2$ is $\mathbb{R}^2$, or the zero subspace, or consists of all scalar multiples of some fixed vector in $\mathbb{R}^2$. (The last type of subspace is, intuitively,
a straight line through the origin.)
	\item Can you describe the subspaces of $\mathbb{R}^3$ ?
\end{enumerate}
\section{Solution}
\begin{enumerate}[label=\alph*.]
\item 
Let $W \neq {0}$ be subspace of $\mathbb{R}^1$. Then $W$ is a nonempty subset of $\mathbb{R}^1$ and there exist $w \in W$ such that $w \neq 0$ which gives us that there exist $w^{-1}$.\\
\\
Let $x\in\mathbb{R}^1$. Since W is in $\mathbb{R}^1$ we have that it is closed under scalar multiplication which gives us that $(xw^{-1})w=x(w^{-1}w)=x.1=x \in W$\\
\\
Hence $\mathbb{R}^1 \subset W$ and therefore $ W = \mathbb{R}^1$\\
\\
Thus the only subspace of $\mathbb{R}^1$ distinct of {0} is $\mathbb{R}^1$ and therefore only subspaces of $\mathbb{R}^1$ are {0} and $\mathbb{R}^1$. 
\\
\item
Clearly, {0} and $\mathbb{R}^2$ itself are subspaces of $\mathbb{R}^2$. If $u \neq 0$ and $u \in \mathbb{R}^2$ then span$\{\vec{u}\}$ = ${c\vec{u}: c\in \mathbb{R}}$ = set of all scalar multiples of $\vec{u}$ is a subspace of $\mathbb{R}^2$.\\
\\
To show that these are the only subspaces of $\mathbb{R}^2$, assume that $W \subset \mathbb{R}^2$ is any subspace of $\mathbb{R}^2$. Since $W\subset\mathbb{R}^2$ is a subspace of $\mathbb{R}^2$, we have that $\vec{0} \in W$. If $W \neq {\vec{0}}$ then there is a vector $\vec{u} \neq 0$ and $\vec{u} \in W$, and hence $W$ contains $c\vec{u}$ for every $c \in \mathbb{R}$. If $W \neq span\{\vec{u}\}$, then there is a vector $v \in W$ so that $\vec{v} \neq k\vec{u}$ for any $k \in \mathbb{R}$.\\
\\
Then $\vec{z} = c\vec{u} + d\vec{v} \in span\{\vec{u}$,$\vec{v}\}$ for any $c,d \in \mathbb{R}$. Since W is a subspace $c\vec{u}$ and $d\vec{v} \in W$ for any $c,d \in \mathbb{R}$, and hence so does $\vec{z} = c\vec{u} + d\vec{v}$. Thus $\vec{z} \in span\{\vec{u}$,$\vec{v}\} \implies z \in W$, and so $span\{\vec{u}$,$\vec{v}\} \subset W \subset \mathbb{R}^2$.\\
\\
Let $\vec{x} = \myvec{x_1 \\x_2} \in \mathbb{R}^2$ be any vector in $\mathbb{R}^2$, and let $\vec{u} = \myvec{u_1\\u_2} \neq \myvec{0\\0}$ and let $\vec{v} = \myvec{v_1\\v_2} \neq \myvec{0\\0}$. We show that there are real numbers $c$ and $d$ so that $c\vec{u} + d\vec{v} = \vec{x}$
\begin{align}
	\label{1}\myvec{cu_1\\cu_2} + \myvec{dv_1\\dv_2} = \myvec{x_1\\x_2}\\
	\label{2}\myvec{u_1 & v_1\\u_2 & v_2}\myvec{c\\d} = \myvec{x_1\\x_2}
\end{align}
Since $\vec{v} \neq k\vec{u}$ for any $k \in \mathbb{R}$ and since $\vec{u} = \myvec{u_1 \\ u_2} \neq \myvec{0\\0}$ assume that $u_1 \neq 0$, and since $k\vec{u} \neq \vec{v} = \myvec{v_1\\v_2} = \myvec{0\\0}$ assume that $v_2 \neq 0$. Then
\begin{align}\label{3}
	A = \myvec{u_1 & v_1\\u_2 & v_2} \rightarrow \myvec{1 & 0\\0 & 1}
\end{align}
\\
Hence $A$ is row equivalent to $I_2$ and so $A$ is invertible and so \eqref{2} has unique solution for $c$ and $d$. Thus for any $\vec{x} \in \mathbb{R}^2$ we can find real numbers $c$ and $d$ such that $\vec{x} = c\vec{u} + d\vec{v}$. Hence $\vec{x} \in \mathbb{R}^2 \implies x \in span\{\vec{u},\vec{v}\}$. Thus $\mathbb{R}^2 \subset span\{\vec{u},\vec{v}\} \subset W \subset \mathbb{R}^2$.\\
\\
Hence $span\{ \vec{u}$,$\vec{v} \}$ = W = $\mathbb{R}^2$, and so the only subspace of $\mathbb{R}^2$ are ${\vec{0}}$, $\mathbb{R}^2$, and $L = {c\vec{u} : \vec{u} \neq 0, c \in \mathbb{R}}$.
\\
\item
The following are the subspaces of $\mathbb{R}^3$ :
\begin{enumerate}[label=\arabic*.]
\item
Origin is a trivial subspace of $\mathbb{R}^3$.\\
\item
$\mathbb{R}^3$ itself is a trivial subspace of $\mathbb{R}^3$.\\
\item
Every line through origin is subspace of $\mathbb{R}^3$.\\
\item
Every plane in $\mathbb{R}^3$ passing through origin is a subspace $\mathbb{R}^3$.\\
\textit{Proof : }
Let $W$ be a plane passing through origin. We need $\vec{0} \in  W$, but we have that since we’re only considering planes that contain origin. Next, we need $W$ is closed under vector addition. If $\vec{w_1}$ and $\vec{w_2}$ both belong to $W$, then so does $\vec{w_1} + \vec{w_2}$ because it’s found by constructing a parallelogram, and the whole parallelogram lies in the plane $W$. Finally, we need $W$ is closed under
scalar products, but it is since scalar multiples lie in a straight line through the origin, and that line lies in $W$. Thus, each plane $W$ passing through the origin is a subspace of $\mathbb{R}^3$.\\
\item
The intersection of any of the above subspaces will also be a subspace of $\mathbb{R}^3$. Because intersection of subspaces of a vector space is also a subspace of vector space.\\
\textit{Proof} :
Let $W$ be a collection of subspaces of $V$, and let $W = \cap W_i$ be their intersection. Since each $W_i$ is a subspace, each of it contains the zero vector. Thus the zero vector is in the intersection $W$, and $W$ is non-empty. Let $\vec{\alpha}$ and $\vec{\beta}$ be vectors in $W$ and let $c$ be a scalar. By definition of $W$, both $\vec{\alpha}$ and $\vec{\beta}$ belong to each $W_i$, and because each $W_i$ is a subspace, the vector $(c\vec{\alpha} + \vec{\beta})$ is again in $W$. Hence by definition of subspace, $W$ is a subspace of $V$.
\\
\end{enumerate}
These 5 are only subspaces of $\mathbb{R}^3$ possible.
\end{enumerate}
\end{document}